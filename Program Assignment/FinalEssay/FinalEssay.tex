\documentclass[10pt]{article}
\usepackage{ctex}
\usepackage{graphicx}
\graphicspath{{./pictures/},{pics/}}%后面接图片的来源
\usepackage{amsmath}
\usepackage{amsfonts}
\usepackage{amssymb}
\usepackage{bm}%用来加粗数学公式
\usepackage{algorithmic}%写有关于算法伪代码相关宏包
\usepackage{algorithm}%写有关于算法伪代码相关宏包
\usepackage{float}%使图片紧跟在文字后面
\usepackage{listings}%用来贴代码
\usepackage{color}
\usepackage{tikz}%用来画图
\newcommand\udl{\underline{\hbox to 6mm{}}}
\renewcommand{\abstractname}{\textbf{\zihao{4}摘\quad 要}}
\title{计算机视觉:AI开眼看世界}
\author{231300027 朱士杭}
\date{\kaishu \today}



\begin{document}
		\maketitle
	\begin{abstract}
		计算机视觉作为人工智能举足轻重的一个分支,在现实生活中已经得到广泛应用,并在未来更多领域有巨大的应用潜力。本文主要探讨机器视觉在AI产业中的应用现状、行业背景、关键技术及其发展趋势,以及对于计算机视觉相关应用的个人思考。首先介绍一些与计算机视觉相关行业的背景,其次分析当下AI潮流中计算机视觉在这些行业应用现状与潜在应用可能,并给出具体实际应用案例,最后阐发一些个人思考,为未来个人发展作铺垫。
		
		\noindent\textbf{关键词:} 计算机视觉、生产生活应用、智能制造
	\end{abstract}
	\section{引言}
		早在AI这股概念兴起之前,计算机视觉作为人工智能的一个重要分支早就已经渗透到人们生活的方方面面。例如安防监控、图像识别与切割、自动泊车等等实际应用,都离不开计算机视觉技术的支持。本文旨在探讨计算机视觉在AI产业中的应用现状、行业背景、关键技术及其发展趋势,并在此基础上提出个人思考。仅供参考,笔者能力有限,若有不足之处,不吝指教。
	\section{行业背景}
	\subsection{计算机视觉概述}
	计算机视觉是一门跨学科的科学领域,它主要研究计算机如何从数字图像或视频中获得高水平的理解。从工程学的角度来看,计算机视觉试图理解和自动化人类视觉系统可以完成的任务。计算机视觉任务包括获取、处理、分析和理解数字图像的方法,以及从现实世界中提取数据以产生数字或符号信息。这种图像理解可以看作借助几何、物理、统计学和学习理论构建的模型,将符号信息从图像数据中分离出来。

	如果说 AI 让计算机能够思考,计算机视觉则让计算机看见、观察和理解。

	计算机视觉的工作原理与人类视觉大致相同,只是人类更胜一筹。人类视觉的优势在于能终身学习各种情境,从而训练如何区分物体、判断与它们的距离、它们是否在移动或某一图像是否存在问题。

	计算机视觉可训练机器来执行这些功能,但它必须在更短时间内使用摄像头、数据和算法而不是视网膜、视神经和视觉皮层来完成这些操作。由于经训练可检查产品或监视生产资产的系统每分钟可分析数千种产品或流程,从而发现难以察觉的缺陷或问题,因此它可以迅速超越人类的能力。

	计算机视觉已用于从能源与公用事业到制造与汽车等各个行业,且该市场的规模还在持续增长。预计到 2022 年,其市场价值将达 486 亿美元。\cite{ref1}

	计算机视觉的基本步骤包括图像获取、前期处理、特征提取、图像分析和解释。图像获取涉及使用传感器或摄像机获取图像或视频数据。前期处理阶段包括对图像进行校正、去噪、增强等操作,以提高后续处理的效果。特征提取是提取图像中的关键特征,如边缘、纹理、颜色等,用于图像的描述和分析。图像分析是通过对特征进行分析和处理,来实现图像分类、目标检测、目标跟踪、场景分割等任务。图像解释则是对图像进行高级推理和理解,如物体识别、场景理解、行为分析等。\cite{ref2}
	
	\subsection{计算机视觉可以应用的行业相关背景}
	计算机视觉相关技术得以应用并非无根之木,无壤之花,而是时代技术发展的需求。一方面,硬件性能的提升和算法的优化,为计算机视觉技术的发展提供了坚实基础;另一方面,互联网和物联网的快速发展积累了海量数据,为计算机视觉模型的训练提供了丰富的资源;更重要的是,作为人肉眼感知的拓展,人的能力达到了一定的上限,生产力已经到了瓶颈期,必然需要更先进效率更高的工具服务于日常生产生活。在当下AI概念迅速崛起的环境之下尤其如此,催生各种行业的变革。甚至可以武断地说,凡是冗杂过剩的工作,均可以解放人类交由计算机完成。
	
	具体针对于计算机视觉而言可以适用的领域主要有两种,一种领域是需要专业知识的专家人才,但是他们也可能犯错,需要精确能力更强的计算机视觉去辅助决策,其次在这些领域里面往往专家人才稀缺,培养周期长,人力成本高,急需计算机视觉工具的应用提高效率,弥补人才岗位的空缺带来的缺陷。
	
	另一种领域就是不需要高精尖人才,但是工作非常冗杂无聊,需要消耗大量人力成本,属于劳动密集型产业,又没有什么技术含量,造成了大量劳动力资源浪费,无论从宏观还是微观角度都一定程度上阻碍了社会经济的发展与进步。而事实上,如果稍微加以计算机(说的好听点就是人工智能)的辅助,将会发生质的飞跃,因为众所周知,计算机最擅长的做的就是重复的工作。
	
	\section{计算机视觉在AI产业中的应用现状}
	综合上面所说的计算机视觉可以应用的行业相关背景分析,在现实生活中,计算机视觉早已经在无人驾驶、监控识别、智能制造、医学影像等等行业得到了广泛的应用,此处将分别进行分析并给出一些实例介绍。
	\subsection{无人驾驶}
	近些年在汽车行业最火的两个概念,一个是新能源,另一个就是无人驾驶技术了。国内外许多企业想要分这块大蛋糕,国外最有名的马斯克手下的特斯拉,前不久还发布了发布会,演示了在实际道路上其智能驾驶可以几乎实现无人操作,并且在开放环境下可以实现安全驾驶,灵活应对各种突发状况;国内武汉前不久落地了无人出租车“萝卜快跑”,做了第一个“吃螃蟹”的人,华为、小米等科技企业也下场着手在无人驾驶领域布局。虽然由于几年前技术不成熟,导致许多事故发生,市场对于无人驾驶前景并不看好,以致于在各造车企业都不敢声称自己为“无人驾驶”,而改称“智能辅助驾驶”,无人驾驶的几个等级从L1到L5大部分车企只敢声称自己才到达了L2阶段或者L2+(因为一旦到了L3倘若发生事故车企就得承担责任)。
	
	在十年前,无人驾驶相关车企仍然以传统激光雷达技术为主,认为计算机视觉等AI算法误差大风险过高。即使近年来AI相关算法发展迅猛,国内车企仍旧对于计算机视觉技术持观望态度,但是无论如何,都不可否认计算机视觉已经在无人驾驶领域得到广泛应用,并且在未来计算机视觉相关技术将是成为主流。特斯拉的无人驾驶技术就是纯粹利用了计算机视觉技术,而摒弃了激光雷达等物理硬件,而是纯粹从软件算法层面优化决策。
	
	这一系列的技术主要通过车载摄像头捕获的图像或视频数据,帮助自动驾驶车辆感知和理解周围环境,包括识别和检测道路标志、交通信号灯、行人、车辆等各种交通参与者。基于深度学习的目标检测算法,如YOLO和Faster R-CNN,使得自动驾驶车辆能够准确地检测和识别道路上的各种目标,并追踪这些目标的运动轨迹,这对于车辆的路径规划、决策制定和行为预测等任务至关重要\cite{ref5}。
	
	车道线检测也是实现自动驾驶的关键步骤之一,通过计算机视觉技术实现对车道线的识别和跟踪,帮助车辆保持在正确的车道上。此外,计算机视觉还在障碍物检测、交通信号识别等方面发挥着重要作用,使得无人驾驶车辆能够遵守交通规则,预测其他交通参与者的行为,并做出相应的反应。
	
	尽管计算机视觉在自动驾驶领域的应用已取得显著成就,但仍需面对多重挑战,包括处理复杂多变的环境场景、整合与互补多种传感器数据,以及确保隐私安全等关键问题,这些都是亟待深入研究和解决的难题。展望未来,随着技术的持续创新和算法的不断优化,计算机视觉在自动驾驶领域的应用潜力将进一步释放。预计计算机视觉技术将更加精确地捕捉和解读周遭环境,为自动驾驶车辆提供更加丰富和可靠的决策支持,从而推动交通行业的深刻变革。
	\subsection{监控识别}
	\subsubsection{交通监控}	
	随着城市化进程的不断加速,交通管理已成为城市管理中不可忽视的重要课题。机动车数量的激增带来了交通需求的增长,导致交通拥堵、事故频发和违章行为等问题日益严重,这些问题不仅影响了市民的出行效率,也对城市的安全和秩序构成了挑战。在这样的行业背景下,计算机视觉技术的引入和应用,为交通监控领域带来了革命性的变革。

	计算机视觉技术通过其高效的数据处理能力和精准的分析手段,为交通监控提供了实时监控和快速响应的能力。它能够自动识别车牌号码,监测车辆速度,甚至分析驾驶员的行为,从而有效预防和减少交通事故的发生。同时,对于违章行为的监控和取证,计算机视觉技术也展现出了极高的效率和准确性,极大地提高了交通执法的公正性和威慑力。此外,政府对于智能交通系统建设的推动,为计算机视觉技术在交通监控领域的广泛应用提供了强有力的政策支持和广阔的市场空间。在这一过程中,智能交通系统的建设不仅提升了交通管理的智能化水平,也为城市交通的可持续发展奠定了坚实的基础。
	
	据我所了解,我们NJUAI有一个与政府合作的项目,关于高速车流量实时检测系统,这也是计算机视觉在交通监控领域的应用,能够实时检测高速路况,汇报事故处理,针对各种情况自动决策提高效率给出预案,给政府相关部门监测管理提供了第一手资料。
	\subsubsection{人脸识别}
	无论是进出小区,亦或是出入宿舍楼,还是打开手机自动解锁……人脸识别早已经渗透了我们日常生活,甚至于现在许多追缉逃犯正是利用人脸识别就抓捕到了逃亡十余年的罪犯。
	
	虽然人脸识别概念早已被人们熟知,但是作为计算机视觉领域最具挑战性的任务之一,其技术也在不断发展和完善。特别是深度学习技术的应用,使得人脸识别的准确性和效率大幅提升。例如,FaceNet、VGGFace等算法已经达到了商业级应用标准,这些算法在安全、支付和智能门禁等领域得到了广泛应用。
	
	但是随着人脸识别相关技术不断成熟,其在个人隐私保护等伦理问题上仍然存在巨大争议,由于本文主要探讨AI应用,此处就不再过多赘述讨论。
	
	至于人脸识别行业的产业链,其大致可分为基础层、技术层和应用层三个部分。基础层位于产业链上游,为行业提供硬件、算法及数据支持。技术层位于产业链中游,主要包括相关人脸识别技术及系统集成,这包括视频人脸识别、图片人脸识别和数据库对比检验等技术。应用层位于产业链下游,具体包括智慧安防、智慧金融、智能交通、移动支付、医疗卫生、政府职能等领域。
	
	智慧安防与智慧金融领域,是人脸识别技术展现其强大实力的主要舞台。在安防领域,人脸识别技术的应用极为普遍,它不仅覆盖了机场、银行、商场等关键公共安全区域,用于门禁系统和监控设备,而且在社区、办公楼等非公共安全场合也发挥着安全监控的重要作用。同样,在金融行业,人脸识别技术也广泛应用于身份认证、移动支付等多个环节。随着技术的持续发展和应用场景的进一步扩展,人脸识别技术的应用架构将变得更加丰富和多样化,满足更广泛的市场需求。
	
	由于技术的不断进步和应用场景的拓展,人脸识别行业市场规模也逐渐扩大,行业的竞争格局也在剧烈变化,越来越多的企业加入了这个领域。在技术方面,一些大型科技公司如谷歌、微软、苹果等都在积极研发人脸识别技术,并取得了显著的进展。同时,一些创业公司也在人脸识别领域取得了突破,国内的企业如腾讯、百度、旷视科技、云从科技、商汤科技等在算法软件方面具有较高的技术实力,已经在全球范围内取得了显著的成绩。随着大企业不断吞并,未来一些专攻于人脸识别的小公司创业企业生存可能会越来越难,但是其他赛道仍然充满了机遇。
	
	\subsection{智能制造}
	计算机视觉在智能制造领域的应用也已经取得了显著的进展,并且未来前景广阔。主要利用计算机和相关设备模拟人类的视觉功能,对实际生产中的图像进行感知、识别和理解。我记得这两年有个概念也很火,叫做“黑灯工厂”,即完全不需要工人的干预,真正意义上实现了全自动化生产,而计算机视觉无疑在其中发挥了重要作用,此时计算机的“眼睛”就替代了人的眼睛,对于一系列生产调整都是依赖于计算机自身,与环境进行交互,自动调整做出反应。
	
	目前,计算机视觉在智能制造中的应用主要集中在产品质检、生产工艺优化、设备预测性维护等方面,这些应用显著提高了制造业流程中的运营效率,并促进了智慧工厂新模式的诞生。机器视觉系统主要用于完成定位、识别、检测、测量等任务,其使用提升了生产线的检测精度、自动化水平、柔性化能力,同时降低了人工成本并保证了工人的人身安全。在半导体及3C电子制造、汽车制造、包装等行业中,机器视觉已有广泛应用,并逐步拓展到医药制造、食品加工、纺织制造等领域。
	
	随着技术的进步,计算机视觉的应用领域不断深化和拓展。深度学习技术的发展极大地增强了计算机视觉在图像识别、场景理解等方面的能力,推动了其在各个领域应用的深化和拓展。例如,智能制造结合了人工智能、大数据、物联网等新一代智能信息技术,贯穿产品全生命周期,实现从工厂到市场的联动。在这一过程中,机器视觉技术作为关键的热门技术之一,以其高度的灵活性、强大的理论支持,在智能制造领域得到了广泛的应用。
	
	据我所了解,我们计算机学院的谢磊教授就曾领导过智能制作项目,与相关制造企业合作开发了基于计算机视觉的高精度实时纠偏系统,用于智能制造行业。2025年在工业生产领域对于IXPE发泡塑料的需求量将达到两千多万吨,这将是一个相当庞大的工业生产量,而在这个生产过程中对于发泡过程的精细控制需要“慧眼”,误差需要控制在0.1mm之内,光靠人眼以及普通的仪器是没有办法达到如此精度,即使是拥有大量丰富经验的“老师傅”都望而却步。而随着该系统的开发应用,可以做到精益求精,精度控制在几十微米之内,这无疑大大增强了企业的生产效率和质量。
	\begin{figure}[H]
		\centering
		\includegraphics[scale=0.32]{cv1}
		\caption{基于计算机视觉的高精度实时纠偏系统}
	\end{figure}
	
	
	\section{计算机视觉关键技术}
	在计算机视觉领域,有五大关键技术,它们分别是图像分类、对象检测、目标跟踪、语义分割和实例分割。因为计算机视觉领域的技术每个点拿出来都可以专门开辟一个新的研究子领域,因而这里仅仅管中窥豹一带而过。
	\subsection{图像分类}
	首先是有关于图像分类技术,在分类过程中要先面对的是图像特征的提取,在深度学习技术尤其是卷积神经网络(CNN)出现之前,这一步骤主要依赖于人工设计的特征,如SIFT(尺度不变特征变换)、HOG(直方图方向梯度)等。这些传统方法虽然在一定程度上能够提取图像的关键信息,但普遍存在泛化能力不足、对复杂场景处理效果不佳等问题。
	
	随着深度学习技术的发展,尤其是CNN的广泛应用,图像分类技术取得了革命性的进步。CNN通过模拟人类视觉系统的层次化处理机制,能够自动从原始像素数据中学习到层次化的特征表示。这一过程无需人工干预,大大简化了特征提取的复杂性。具体来说,CNN的基本结构包括多个卷积层和池化层,以及最后的全连接层。卷积层通过卷积操作提取图像中的局部特征,如边缘、角点、纹理等;池化层则负责降低特征的空间维度,同时保留重要信息;全连接层则将这些特征组合起来,进行分类决策。
	
	为了实现高效的图像分类,CNN需要在大规模标记图像数据集上进行训练\cite{ref6},对于这些数据集,如ImageNet,包含了数百万张图像和数千个类别。通过这样的训练,CNN能够学习到丰富的特征表示,这些特征对于区分不同类别的图像至关重要。
	
	在实际应用中,图像分类技术已经取得了显著成果。例如,在图像检索领域,用户可以通过上传一张图片,快速找到与之相似的其他图片;在社交媒体过滤方面,图像分类技术可以帮助平台自动识别和过滤不当内容;在广告投放领域,则可以根据图像内容智能推荐相关广告,提高广告的投放效果。
	\subsection{对象检测}
	其次是对象检测技术,计算机不仅要识别图像中的对象是什么,还要确定对象在图像中的具体位置。这项技术的挑战性在于,它需要在复杂多变的图像背景中,准确地识别出目标对象,并精确地描绘出其边界框。随着深度学习技术的突破,对象检测算法已经取得了显著的进展,成为当今智能系统不可或缺的一部分。
	
	传统的对象检测方法往往依赖于手工设计的特征和滑动窗口技术,这种方法计算量大,且在处理复杂场景时效果不佳。而深度学习算法,尤其是卷积神经网络(CNN)的出现,极大地改变了这一局面。
	基于Region Proposal的算法,如Faster R-CNN,首先通过一个区域建议网络(RPN)来生成一系列可能包含目标的候选区域,然后对这些区域进行分类和边界框回归。Faster R-CNN通过共享卷积特征图,提高了检测效率,同时引入了锚点(anchor)机制,使得算法能够处理不同尺度和比例的目标。
	
	You Only Look Once(YOLO)系列算法则采用了完全不同的思路。YOLO将对象检测视为一个单一的回归问题,直接在图像上预测边界框和类别概率。YOLO的优势在于其极高的检测速度,使其适用于需要实时检测的应用场景。从YOLO到YOLOv4,算法通过引入新的网络结构、损失函数和训练技巧,不断优化检测性能。
	
	Single Shot MultiBox Detector(SSD)则是另一种流行的对象检测算法。SSD算法通过在特征图上均匀地采样多个默认框,并预测这些框的类别和偏移量,实现了在单次前向传播中同时进行分类和定位。SSD算法在保持较高检测速度的同时,也取得了不错的检测精度。
	
	\subsection{目标跟踪}
	目标跟踪则是致力于在视频序列中精确地追踪目标的位置和运动轨迹,这一技术在视频分析、运动行为理解、无人驾驶等多个领域扮演着至关重要的角色。近年来,随着深度学习技术的迅猛发展,尤其是卷积神经网络(CNN)的应用,目标跟踪技术取得了显著的突破。这些基于深度学习的跟踪算法通过从视频帧中提取目标的特征表示,采用在线学习的方式不断优化和更新跟踪模型,从而显著提升了跟踪的准确性和鲁棒性。在这一过程中,算法能够适应目标的外观变化、光照条件、遮挡等复杂情况,使得跟踪系统能够在多种环境下保持稳定的表现
	
	目标跟踪的方法主要分为2大类,一类是相关滤波、一类是深度学习。相比于光流法、Kalman、Meanshift等传统算法,相关滤波类算法跟踪速度更快,深度学习类方法精度高。另外,具有多特征融合以及深度特征的追踪器在跟踪精度方面的效果更好,使用强大的分类器是实现良好跟踪的基础,而尺度的自适应以及模型的更新机制也影响着跟踪的精度。
	
	在此基础上还演化出了另一种技术多目标跟踪,一般简称为MOT(Multiple Object Tracking),也有一些文献称作MTT(Multiple Target Tracking)。在事先不知道目标数量的情况下,对视频中的行人、汽车、动物等多个目标进行检测并赋予ID进行轨迹跟踪。不同的目标拥有不同的ID,以便实现后续的轨迹预测、精准查找等工作\cite{ref4}。这里就不再进行过多展开介绍。
	
	我记得计算机学院那边的MCG研究组就有过相关研究,关于如何对视频进行切割以及目标跟踪的相关研究,此处展示一下相关技术研究。
		\begin{figure}[H]
		\centering
		\includegraphics[scale=0.33]{mcg}
		\caption{南京大学MCG媒体计算研究组有关目标跟踪技术研究\cite{ref3}}
	\end{figure}
	
	\section{一些个人思考}
	计算机视觉的应用前景其实非常广泛,虽然这几年风头好像都被大语言模型抢走了,而且相关理论研究似乎到了一个瓶颈期,很难有质的突破,但是毋庸置疑的是,在实际应用中,计算机视觉反而应用是最广泛的,或者说是最“实用”的。如果说即使未来不从事理论方面的研究,在实际应用方面仍然可以大有作为。虽然在当下后疫情时代经济下行压力巨大,各个赛道上大家都在为存量厮杀竞争,但是学到AI知识技术之后毕业的时候仍然可以寻找增量。
	
	作为理工科学生,可能在大部分人的刻板印象里就是一味埋头学习,两耳不听窗外事。但是要时刻保持着清醒,我们现在所学的一切将来都是要为了应用,我们所学的所有的理论都要为了应用而服务。多出去走走,其实机会就在我们的身边。
	
	现在有很多人属于要么有想法,但是没有技术,或者说因为缺乏技术,而不清楚自己想法的可行性的“空想家”,要么就是属于空有技术但是没有想法的“工具人”。作为NJUAI的学生我们必须得既要有自己的想法又要有扎实的技术(修炼好内功的Leader,这点我们还是得做到的),这样走入社会之后无论是就业还是创业,我们都可以波澜不惊。说实话同样是AI专业出身的,当学到一定境界的时候,大家所掌握的知识技术水平都大差不差,那我们之间的区别就在一些其他能力了,也就是我们通常所说的综合能力。
	
	所谓的综合能力包括但不限于团队协作能力,领导能力,思维模式等等一系列的能力,避免成为一个工具人,之后区分我们的就是想法层面的,如果一直就业,中年危机裁员之后何去何从?计算机相关行业出身的不可能打一辈子的工,更新迭代太快了,我们迟早会被公司淘汰,那么如何生存?掌握了技术是远远不够的,我们需要有一个好的想法,而想法正是从日常生活中得来,我在这篇文章中分析的所有有关于计算机视觉的应用,其实已经有很多人在做了,那我们还需要重复造轮子吗?其实是有的,因为市场上竞争的不只是几家企业,与相关企业合作的也不是只有几家,还有更多的企业需要相关技术的应用。
	
	但是实际情况呢?就以我自己为例,我们家是做生意的,而在他们的市场里普遍认为,即使当下AI非常火爆,AI离普通人还是太远了,真正的应用还轮不到他们。这就属于下沉市场啊,这也是未来大有可为的一个领域,自己有跟他们其中一些人聊过,他们的需求其实挺简单的,虽然他们可能觉得这个技术非常高大上,但是从一个拥有一些知识技术的人的角度来说,其实实现非常容易,但是没有人去做,没有人去关心,大家的眼光都放在了目前火热的领域中去了,没有人去关心下沉市场的需求(蚊子再小也是肉啊)。但是事实上,只有真正用AI技术解决了下沉市场中的一些难点痛点,所谓的“新质生产力”才有它存在的意义。
	
	说的好像有点远了?似乎和计算机视觉没有什么关系?其实不是的,在本文中对计算机视觉分析了这么多主要就是要说明一点:“它广泛应用于我们日常生产生活实践中”,仅此而已,而我刚刚说了这么多,其实既想告诉读者也想告诉自己,如果从事计算机视觉方向工作研究,一定要学会“开眼看世界”,时刻保持着清醒,自己身边的机会非常多,只是缺乏一双发现机会的眼睛。如果在某一个你突然发现,这个事情是不是可以不用人去做?那么恭喜你,你的机会来了。
	
	
	
	
	
	
	
	
	
	\begin{thebibliography}{99}  
		\bibitem{ref1}  7 Amazing Examples Of Computer And Machine Vision In Practice, Bernard Marr, Apr 8, 2019
		\bibitem{ref2} 计算机视觉(和卷积神经网络)简史,Rostyslav Demush,Hacker Noon,2019 年 2 月 27 日
		\bibitem{ref5} 王小禹.无人驾驶环境中视觉感知关键技术研究[D].长春工业大学,2022.DOI:10.27805/d.cnki.gccgy.2022.000448.
		\bibitem{ref6} 罗建豪,吴建鑫.基于深度卷积特征的细粒度图像分类研究综述[J].自动化学报,2017,43(08):1306-1318.DOI:10.16383/j.aas.2017.c160425.
		\bibitem{ref4} 基于密度图多目标追踪的时空数据可视化, 宋思程, 陈辰, 李晨辉, 王长波
		\bibitem{ref3} USER-ORIENTED STEREO VIDEO REFOCUSING BY COMPUTATIONAL
		CINEMATOGRAPHIC MODEL Wenjing Geng, Dapeng Du, Tongwei Ren and Gangshan Wu, State Key Laboratory for Novel Software Technology, Nanjing University, Nanjing 210023, China
		
	\end{thebibliography}
\end{document}